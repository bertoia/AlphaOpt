\def\year{2017}\relax

\documentclass[letterpaper]{article}
\usepackage{aaai17}
\usepackage{times}
\usepackage{helvet}
\usepackage{courier}
\usepackage{amsmath}
\usepackage{amsfonts}
\usepackage{multirow}
\frenchspacing
\setlength{\pdfpagewidth}{8.5in}
\setlength{\pdfpageheight}{11in}
\pdfinfo{
/Title (Bayesian Optimisation of Machine Learning Algorithms)
/Author ()}
\setcounter{secnumdepth}{1}
 \begin{document}
% The file aaai.sty is the style file for AAAI Press
% proceedings, working notes, and technical reports.
%
\title{Bayesian Optimisation of Machine Learning Algorithms}
\author{
National University of Singapore \\
CS4246 Group 7 \AND
\normalsize\normalfont\textbf{Cheong, Xuan Hao Filbert A0121261Y} \\
\normalsize\normalfont\textbf{Gay, Ming Jian Davis A0111035A} \\
\normalsize\normalfont\textbf{Karen Ang Mei Yi A0116603R} \And
\normalsize\normalfont\textbf{Quek, Chang Rong Sebastian A0121520A} \\
\normalsize\normalfont\textbf{Vincent Seng Boon Chin A0121501E} \\
\normalsize\normalfont\textbf{Xu, Ruofan A0100965J}
}

\maketitle
\begin{abstract}
Machine learning algorithms have grown in popularity over the past few years.
This increased interest and usage also implies an increasing need for
hyperparameter tuning. Traditionally done by human experts, the process is
nonetheless a black art in itself. Experts often use heuristics to guide their
tuning efforts. This is not invulnerable to cognitive biases. Moreover, humans
are unlikely to find a result close to the optimum, especially if there are
numerous hyperparameters and when hyperparameters are correlated in some way
not easily inferred. In this paper, we explore the use of Gaussian Processes to
automate this problem through the use of Bayesian Optimisation.
\end{abstract}

\section{Introduction}
\noindent Machine learning algorithms often involves careful tuning of hyperparameters.
As an example, the hyperparameters of a neural network include the learning rate,
batch size and number of hidden units. Given the black-box nature of this algorithm,
there is no definite and explicit method to select the optimal hyperparameters.
Not only is the performance of such algorithms particularly sensitive to the
parameters involved, this process of hyperparameter optimisation is traditionally
done by a human, who is unlikely to find the optimum easily. The parameters are
also often continuous in nature, making this a difficult optimization problem for
machine learning practitioners.

Currently, the two common strategies employed for tuning are grid search and
random search. The former involves the uniform division of the entire search
space. This is easily seen with a simple example of two parameters, $\theta_1$ and
$\theta_2$, which are allowed to vary along the continuous domain of $(0, 1)$.
Suppose we divided the search space for each parameter at a step interval of 0.5,
that is, the set of values: $\Theta=\{0.0, 0.5, 1.0\}$. Then, the points that one
would evaluate the algorithm at would be given by the cartesian product: $\Theta \times \Theta$.
The second strategy simply involves the random selection of a number of points
within the search space. Notably, not only is this strategy easy to implement, it
often performs comparably better than grid search.

\section{Motivating Application}
Machine learning algorithms are widely used in academia and industry in learning
problems such as classification and regression. Automating hyperparameter optimisation
can better improve the performance of these algorithms, which can indirectly lead
to greater scientific or industrial advances and innovations.

Current tuning strategies outlined above has their limitations. Grid search is essentially
conducting an exhaustive search in the parameter space and thus potentially expensive,
especially when the parameter space is continuous and has a large domain size. For
grid search, this problem is usually mitigated by increasing search step interval
or narrowing parameter space. For example, in tuning of Support Vector Machines
(SVM) with RBF kernel, it is recommended that $\textit{C}$ and $\gamma$ are searched
exponentially in base 2: $\textit{C}=\{2^{-5},2^{-3},...,2^{15}\}$ and
$\gamma=\{2^{-15},2^{-13},...,2^{2}\}$. A more economical approach could be to search
$\textit{C}$ exponentially in base 10 for a coarser search
($\textit{C}=\{10^{-5},10^{-3},...,10^{15}\}$) and searching over a smaller domain
exponentially in base 2 for $\gamma$ ($\gamma=\{2^{-6},2^{-4},...,2^{2}\}$). For random
search, while it is more effective than grid search epecially in high-dimensionsal spaces,
it is stochastic in nature and termination conditions (maximum iterations or threhold change)
has to be specified. Therefore the trade-off between model performance and time needed
for tuning is arbitrarily determined.

In light of the abovementioned problems with conventional strategies, it would be ideal
to have an algorithm that allow us to mathemathically formuate the exploit and explore
trade-off in hyperparameter optimization to train the machine learning model the least
amount of times on parameters that, on each training evaluation instance, acquire the
most information (the concept of "infomation", known as criterion will have to be defined,
as seen later). This can enable the model to reach the near optimum with as little trials as possible.
It is with this motivation that we approach hyperparameter tuning through Bayesian optimisation.

Bayesian optimisation involves the use of a surrogate function (or response surface)
which approximates the objective. For a certain number of iterations, or until
convergence, we pick, according to a heuristic or function, henceforth known as the
acquisition function, the next best point to be evaluated by the objective. The
surrogate function is updated in this process to reflect our new belief.

In particular, the Gaussian Process model is used as the surrogate function
for the objective function, which in this case, is the machine learning algorithm.
The Gaussian Process model, being a distribution over functions, is especially apt
to model this unknown, non-convex, and complex objective function. The
initial Gaussian Process model prescribes a prior over the objective, which is
subsequently updated with new data that is observed.

This differentiates the strategy of Bayesian optimisation from other tuning
strategies mentioned previously. Instead of discarding information about new points
that we have queried the objective for, Bayesian optimisation cleverly exploits this
knowledge by updating the prior. As a result, this can provide improvements in time
efficiency over naive algorithms such as grid search, whereby the computational
cost scales exponentially with the number of parameters.  This is particulary
important in our application where the search space is usually high-dimensional
and extremely large. Since the Gaussian Process model provides predictive uncertainty
in the posterior, we can exploit this in the acquisition function, which essentially
computes the utility of candidate points and is optimised to obtain the next point
to be evaluated.

Inherent in this process is this tradeoff between exploration and exploitation.
Clearly, one could simply greedily select the next point with the highest predictive
mean, or a point which gives the highest probability of improvement. However, one
is also likely to fall into local minima or maxima in this case. Without exploration
in areas of high uncertainty, it is not guaranteed that the global optimum can be
found. This tradeoff is usually manifest in the acquisition function. In fact,
we will explore various Bayesian optimisation criteria and conduct experiments to
evaluate their performance in our application.

Finally, there are certain Bayesian optimisation algorithms that can satisfy the
requirements of hyperparameter tuning. Firstly, the tuning process is often
expensive. Evaluations of large models dealing with big data can be extremely
costly. Moreover, it is innately useful and beneficial if we can utilise
Bayesian optimisation to improve this search for the optimum. In other contexts,
one may be limited by time or computing resources. Consequently, minimising the
number of function evaluations to obtain a near-optimal result is clearly a concern.
Conversely, one may have access to parallel computing resources. In this aspect,
Bayesian optimisation is flexible in that both sequential and parallel algorithms
are available. In particular, non-myopic and parallel Bayesian optimisation
algorithms can effectively reduce the number of objective function evaluations
as well as the time needed to optimise the objective.

\section{Technical Approach}
% Describe clearly and CONCISELY the technical details of the active learning or Bayesian optimization algorithm that you are using.

\subsection{Problem Definition}
An optimization problem typically involves the finding of a minimum f(x) by exploiting a model to pick the next point to evaluate the function. As training a machine learning algorithm can be expensive, it is worthwhile to invest more computational cost in picking a promising set of hyperparameters to evaluate next. In this paper, we choose a Gaussian Process model as our prior and an acquisition function to pick the next point to evaluate by constructing a utility function from the model posterior.

Besides the number of trials, we also take consideration of the time for each trial and pick the point that not only yield good result but fast to evaluate.

\subsection{Model Definition}
In this paper, we assume that the performance measures for hyperparameters follow a Gaussian distribution. Given n set of hyperparameters which have been evaluated, we predict the performance measure of the next set.

In the cost-sensitive model, we maintains another GP model to predict the execution time given the same input.

\subsection{Acquisition Functions}

\subsection{Covariance Functions}
The GP function is mainly characterized by its covariance function after normalizing the data to attain a mean of 0. The covariance function produces a covariance matrix which is utilised by the Gaussian process model for inference.
We propose the use of the ARD Mat´ern 5/2 kernel
\begin{align*}
	k (\textbf{x},\textbf{x}') = \sigma^2(1+\sqrt5r+\frac{5}{3}r^2)\exp(-\sqrt5r)
\end{align*}

The Mat'ern 5/2 kernel is commonly used in the areas of machine learning and multivariate statistical analysis on metric spaces. Due to its stationary and isotropic nature, it is able to produce similar predictive output for two inputs that are d units distant from each other. 5/2 is chosen for its ability to produce smooth and yet sensitive curves.

\subsection{Cost-sensitive Modeling}

While it is important to find a good set of hyperparameters in the fewest number of trials, we should keep in mind that the execution time for different regions of the parameter spaces can vary significantly. In the situation of limited budget, such as time or money, we propose a cost-sensitive model to select the next point based on expected improvement per second. A separate GP model is used to model the evaluation time of each setting of hyperparameters. By dividing the expected improvement of each point by its cost, the new acquisition function selects points that balance between good performance and cheap evaluation cost.

\section{Experimental Setup}

\subsection{Procedure}

\section{Experimental Evaluation}

\section{Improvements}

\section{References}

\section{Roles and Contributions}

\begin{description}
\item [Cheong Xuan Hao Filbert]
\item [Gay Ming Jian Davis]
\item [Karen Ang Mei Yi]
\item [Quek Chang Rong Sebastian]
\item [Vincent Seng Boon Chin]
\item [Xu Ruofan]
\end{description}

\end{document}
